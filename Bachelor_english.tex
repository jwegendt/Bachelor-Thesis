%%%%%%%%%%%%%%%%%%%%%%%%%%%%%%%%%%%%%%%%%%%%%%%%%%%%%%%%%%%%%%%%%%%%%%%%%%%%%
%%% LaTeX-Rahmen fuer das Erstellen von englischen Bachelorarbeiten
%%%%%%%%%%%%%%%%%%%%%%%%%%%%%%%%%%%%%%%%%%%%%%%%%%%%%%%%%%%%%%%%%%%%%%%%%%%%%

%%%%%%%%%%%%%%%%%%%%%%%%%%%%%%%%%%%%%%%%%%%%%%%%%%%%%%%%%%%%%%%%%%%%%%%%%%%%%
%%% allgemeine Einstellungen
%%%%%%%%%%%%%%%%%%%%%%%%%%%%%%%%%%%%%%%%%%%%%%%%%%%%%%%%%%%%%%%%%%%%%%%%%%%%%

\documentclass[twoside,12pt,a4paper]{article}
\usepackage[utf8]{inputenc}
%\usepackage{reportpage}
\usepackage{epsf}
\usepackage{graphics, graphicx}
\usepackage{latexsym}
\usepackage[margin=10pt,font=small,labelfont=bf]{caption}

\usepackage{amsmath}
\usepackage{array} %for the \newcolumntype macro
\usepackage{amsthm} %for theorems/definitions
\usepackage{thmtools} %for listofheorems TODO: remove/integrate?


\textwidth 14cm
\textheight 22cm
\topmargin 0.0cm
\evensidemargin 1cm
\oddsidemargin 1cm
%\footskip 2cm
\parskip0.5explus0.1exminus0.1ex

% Kann von Student auch nach persönlichem Geschmack verändert werden.
\pagestyle{headings}

\sloppy

\usepackage{hyperref}

\begin{document}


%%%%% Macros %TODO: sollten die wo anders hin/in ein eigenes file?


%Macros for Theorems/Definitions/...
\theoremstyle{definition}
\newtheorem{definition}{Definition}[section]
\newtheorem{theorem}{Theorem}[section]

%Macros for possible terms in ND:
\newcommand{\constructor}{K(\overline{e})}
\newcommand{\destructor}{e.d(\overline{e}) }
\newcommand{\patmatch}{e.\textbf{case}\{\overline{K(\overline{x})\Rightarrow e}\}} 
\newcommand{\copatmatch}{\textbf{cocase}\{\overline{d(\overline{x})\Rightarrow e}\}}

%Macros for terms but for examples and such
\newcommand{\exconstructor}[1]{K(#1)}
\newcommand{\exdestructor}[1]{e.d(#1) }
\newcommand{\expatmatch}[1]{e.\textbf{case}\{#1\}} 
\newcommand{\excopatmatch}[1]{\textbf{cocase}\{#1\}}

%Macros for structural/technical latex stuff
\newcolumntype{L}{>{$}l<{$}} % a left-aligned column in mathmode
\newcolumntype{C}{>{$}c<{$}} % a center-aligned column in mathmode

\newcommand{\unifvar}{\alpha^?}
\newcommand{\FV}[1]{\text{FV}(#1)}
\newcommand{\listofexpr}{e_1,...e_n}
\newcommand{\listofvar}{x_1,...x_n}
\newcommand{\betaconv}{\equiv_{\beta}^1}
\newcommand{\etaconv}{\equiv_{\eta}^1}
\newcommand{\Ap}[1]{\texttt{Ap}(#1)}

\newcommand{\partition}{\;|\;}

%%%%%%%%%%%%%%%%%%%%%%%%%%%%%%%%%%%%%%%%%%%%%%%%%%%%%%%%%%%%%%%%%%%%%%%%%%%%
%%% Layout Title page
%%%%%%%%%%%%%%%%%%%%%%%%%%%%%%%%%%%%%%%%%%%%%%%%%%%%%%%%%%%%%%%%%%%%%%%%%%%%
 
\begin{titlepage}
 \begin{center}
  {\LARGE Eberhard Karls Universität Tübingen}\\
  {\large Mathematisch-Naturwissenschaftliche Fakultät \\
Wilhelm-Schickard-Institut für Informatik\\[4cm]}
  {\huge Bachelor Thesis Computer Science\\[2cm]}
  {\Large\bf  Higher-Order Unification for Data and Codata Types\\[1.5cm]}
 {\large Julia Wegendt}\\[0.5cm]
Date\\[3cm]
{\small\bf Reviewer}\\[0.5cm]
 {\large Prof. Dr. Klaus Ostermann}\\
  {\footnotesize Department of Computer Science\\
	University of Tübingen}
\end{center}
	
\end{titlepage}
%%%%%%%%%%%%%%%%%%%%%%%%%%%%%%%%%%%%%%%%%%%%%%%%%%%%%%%%%%%%%%%%%%%%%%%%%%%%
%%% Layout back of title page
%%%%%%%%%%%%%%%%%%%%%%%%%%%%%%%%%%%%%%%%%%%%%%%%%%%%%%%%%%%%%%%%%%%%%%%%%%%%

\thispagestyle{empty}
\vspace*{\fill}
\begin{minipage}{11.2cm}
\textbf{Wegendt, Julia}\\
\emph{Higher-Order Unification for Data and Codata Types}\\ Bachelor Thesis Computer Science\\
Eberhard Karls Universität Tübingen\\
Period: from-till
\end{minipage}
\newpage

%%%%%%%%%%%%%%%%%%%%%%%%%%%%%%%%%%%%%%%%%%%%%%%%%%%%%%%%%%%%%%%%%%%%%%%%%%%%

\pagenumbering{roman}
\setcounter{page}{1}

%%%%%%%%%%%%%%%%%%%%%%%%%%%%%%%%%%%%%%%%%%%%%%%%%%%%%%%%%%%%%%%%%%%%%%%%%%%%
%%% Abstract
%%%%%%%%%%%%%%%%%%%%%%%%%%%%%%%%%%%%%%%%%%%%%%%%%%%%%%%%%%%%%%%%%%%%%%%%%%%%

\section*{Abstract}

Write here your abstract.
\newpage
%%%%%%%%%%%%%%%%%%%%%%%%%%%%%%%%%%%%%%%%%%%%%%%%%%%%%%%%%%%%%%%%%%%%%%%%%%%%
%%% Acknowledgements
%%%%%%%%%%%%%%%%%%%%%%%%%%%%%%%%%%%%%%%%%%%%%%%%%%%%%%%%%%%%%%%%%%%%%%%%%%%%
\section*{Acknowledgements}

Write here your acknowledgements.

\cleardoublepage

%%%%%%%%%%%%%%%%%%%%%%%%%%%%%%%%%%%%%%%%%%%%%%%%%%%%%%%%%%%%%%%%%%%%%%%%%%%%%
%%% Table of Contents
%%%%%%%%%%%%%%%%%%%%%%%%%%%%%%%%%%%%%%%%%%%%%%%%%%%%%%%%%%%%%%%%%%%%%%%%%%%%%

\renewcommand{\baselinestretch}{1.3}
\small\normalsize

\tableofcontents

\renewcommand{\baselinestretch}{1}
\small\normalsize

\listoftheorems

\cleardoublepage

%%%%%%%%%%%%%%%%%%%%%%%%%%%%%%%%%%%%%%%%%%%%%%%%%%%%%%%%%%%%%%%%%%%%%%%%%%%%%
%%% Main Part
%%%%%%%%%%%%%%%%%%%%%%%%%%%%%%%%%%%%%%%%%%%%%%%%%%%%%%%%%%%%%%%%%%%%%%%%%%%%%

\pagenumbering{arabic}
\setcounter{page}{1}

\section{Introduction}

%TODO

\section{The Untyped Calculus ND}

I will be introducing the Untyped Calculus ND, based on ... %TODO citation

\subsection{Syntax of the Untyped Calculus ND}

%TODO: introduction

%% explanation of lists

Some knowledge of notation is necessary to familiarize oneself with the syntax of the Untyped Calculus.
$X$ represents a (possibly empty) sequence $X_1, ... X_i, ... X_n$.  
%TODO: das mit dem simultaneos indexing besser verstehen
%TODO: das mit dem X_1, ... 
%TODO: will ich variablen und co auch hier einführen?
\\
A pattern match $e.\textbf{case}\{\overline{K(\overline{x})\Rightarrow e}\}$ 
matches a term $e$ against a sequence of clauses, each clause consisting of a constructor and an expression.
The expression associated with first constructor %TODO: oder kann es nur eine geben?  
that matches the term is the result of the pattern match. %TODO: kann man hier überhaupt von result sprechen?
For a copattern match $\textbf{cocase} \{\overline{d(\overline{x}) \Rightarrow e}\}$
the same rules apply, but instead of constructors, the term is matched against destructors.

\iffalse
\begin{equation}
    \begin{split}
        e ::&= x \quad \text{Variables} \\
        &| \; \constructor       \quad \text{Constructors} \\
        &| \; \destructor         \quad \text{Destructors} \\
        &| \; \patmatch       \quad \text{Pattern match}\\
        &| \; \copatmatch      \quad \text{Copattern match}
    \end{split}
\end{equation}
\fi

\begin{definition}[Terms of the Calculus ND]

\begin{table}[!h]
\centering
    \begin{tabular}{L C L l}
        e & ::= & x & Variable \\
          & | & \constructor & Constructor \\
          & | & \destructor & Destructor\\
          & | & \patmatch & Pattern match\\
          & | & \copatmatch & Copattern match
    \end{tabular}
\end{table}
\end{definition}


\iffalse
\begin{align*}
    e  ::&=  x  \tag*{Variable} \\
        &| \quad \constructor \tag*{Constructor} \\
        &| \quad \destructor  \tag*{Destructor}\\
        &| \quad \patmatch  \tag*{Pattern match}\\
        &| \quad \copatmatch  \tag*{Copattern match}
\end{align*}
\fi


In pattern and copattern matches, every con- or destructor may occur no more than once.
%TODO: erklären warum? => verstehen warum (das reicht)?

Let's look at some examples to ... %TODO: Formulierung 
All rudimentary datatypes can be constructed through constructors: %TODO: zu blanket statement/stimmt das wirlich? 
\\
The Booleans \texttt{True} and \texttt{False} are constructors on an empty sequence.\\
%TODO: integers müssen wir genau so machen, oder?
%TODO: gibt es noch andere interessante/einleuchtende beispiele?
%TODO: Idee für Strings: wir machen das gleiche mit buchstaben wie bei booleans 
%      und representieren dann strings als konstruktoren die eine liste an buchstaben nehmen?

Abstract data types like lists, arrays, records etc. can similarly be defined through constructors:\\
\texttt{Cons(True, Cons(False, Nil))} and \texttt{Date(...)} %TODO: wie werden nummern repräsentiert?

We can use Pattern matching to represent conditionals:
Lik this if statement: "$\textbf{if} e_1 \textbf{then} e_2 \textbf{else} e_3$" which is analogous to:
$e_1.\textbf{case}\{\texttt{True}\Rightarrow e_2, \texttt{False}\Rightarrow e_3\}$
Or the expression which tests whether a given list contains the number 5:
$e.\textbf{case}\{\}$

\subsection{Free Variables, Substitutions, Contexts}

\begin{definition}[Free Variables]
    The set of free variables of a term $e$ is FV($e$). A term is closed if this set is empty.
    Free Variables are defined recursively over the structure of terms as follows:
    \begin{align*}
        \begin{split}
            \FV{x} &:= \{x\}\\
            \FV{\exconstructor{\listofexpr}} &:= \FV{e_1} \cup \dots\cup \FV{e_n}\\
            \FV{\exdestructor{e_1,...e_n}} &:= \FV{e}\cup FV(e_1) \cup \dots \cup \FV{e_n}\\
            \FV{\patmatch} &:= \FV{e}\cup (\FV{e_1}\backslash\overline{x})\cup \dots\cup (\FV{e_n}\backslash\overline{x})\\
            \FV{\copatmatch} &:= (\FV{e_1}\backslash\overline{x})\cup \dots\cup (\FV{e_n}\backslash\overline{x})
        \end{split}
    \end{align*}
%TODO: explanation
\end{definition}


\begin{definition}[Substitution]
    A simultaneous substitution $\sigma$ of the terms $\listofexpr$ for the distinct variables $\listofvar$ is defined as follows:
    \begin{align*} %TODO: geht das mit dem distinct da oben unter?
        \sigma ::= [\listofexpr \backslash\listofvar]
    \end{align*}
\end{definition}

The set of variables for which the substitution is defined is called the domain.
The set of free variables which appear in the substitution is called the range. %TODO: formulierung ausschmücken?

\begin{definition}[Domain and Range of a Substitution]
    The definitions of Domain and Range of a Substitution are as follows:
    \begin{align*}
        \texttt{dom}([\listofexpr/\listofvar]) &:= \{\listofvar\}\\
        \texttt{rng}([\listofexpr/\listofvar]) &:= \FV{e_1}\cup \dots\cup\FV{e_n}
    \end{align*}
\end{definition}

What is actually interesting is what happens when we appy a substitution to an expression

\begin{definition}[Action of a Substitution]
    The action of a substitution $\sigma$ on a term $e$, written as $e \sigma$ and is defined as follows:
    \begin{align*}
        x[\listofexpr / \listofvar] :=& e_i \quad(\text{if } x=x_i) \\
        y\sigma :=& y\quad (\text{if }y\notin \texttt{dom}(\sigma))\\
        (\exconstructor{\listofexpr})\sigma :=& \exconstructor{e_1 \sigma,\dots , e_n \sigma}\\
        (\exdestructor{\listofexpr}) \sigma:=& (e \sigma).d(e_1 \sigma,\dots , e_n \sigma)\\
        (\patmatch)\sigma:=& (e\sigma).\textbf{case}\{\overline{K(\overline{y}\Rightarrow (e\sigma')\sigma)}\}\\
        (\copatmatch)\sigma:=& \excopatmatch{\overline{(d(\overline{y})\Rightarrow(e\sigma')\sigma)}}
    \end{align*} 
    Where $\sigma'$ is a substitution that ensures that we don't bind new variables: 
    $\sigma'$ has the form $[y_1,\dots, y_n/\listofvar]$ and all $y_i$ are fresh for both the domain and the range of $\sigma$.
\end{definition}

The composition of two substitutions $\sigma_2 \circ \sigma_1$ which is equivalent to first applying the substitution $\sigma_1$, then the substitution $\sigma_2$.

\begin{definition}[Composition of Subsitutions]
    Given two substitutions
    \begin{align*}
        \sigma_1 := [\listofexpr/\listofvar],\qquad \sigma_2 := [t_1,\dots,t_m/y_1,\dots,y_m]
    \end{align*}
    Composition is defined as:
    \begin{align*}
        \sigma_2\circ\sigma_1 := [e_1\sigma_2,\dots,e_n\sigma_2,t_j,\dots,t_k/\listofvar,y_j,\dots,y_k]
    \end{align*}
    Where $j, \dots, k$ ist the greatest sub-range of indices $1,\dots,m$ such that none of the variables $y_j$ to $y_k$ is in the domain of $\sigma_1$ 
    %TODO: kann man das besser formulieren?
\end{definition}

\begin{definition}[Idempotency]
    A substitution $\sigma$ is idempotent, iff. $\sigma \circ \sigma = \sigma$.
    Concretely, this means that it doesn't matter how often we apply a substitution to a given problem.
\end{definition}

For example, $[\excopatmatch{\Ap{x}\Rightarrow x} / y]$ is idempotent, since:
\begin{align*}
    &[\excopatmatch{\Ap{x}\Rightarrow x} / y] \circ [\excopatmatch{\Ap{x}\Rightarrow x} / y] \\
    &= [\excopatmatch{\Ap{x}\Rightarrow x}[\excopatmatch{\Ap{x}\Rightarrow x} / y] / y] \\
    &= [\excopatmatch{\Ap{x}\Rightarrow x} / y] 
\end{align*}

On the other hand, the substitution $[\excopatmatch{\Ap{y}\Rightarrow x} / x]$ is not idempotent, since:
\begin{align*}
    &[\excopatmatch{\Ap{y}\Rightarrow x} / x] \circ [\excopatmatch{\Ap{y}\Rightarrow x} / x]\\ 
    &= [\excopatmatch{\Ap{y}\Rightarrow x}[\excopatmatch{\Ap{y}\Rightarrow x} / x] / x] \\
    &= \excopatmatch{\Ap{y}\Rightarrow (\Ap{y}\Rightarrow x) } \neq [\excopatmatch{\Ap{y}\Rightarrow x} / x]
\end{align*}

\begin{definition}[More General] 
    A substitution $\sigma$ is more general than a substitution $\theta$, iff. there exists a mapping $\tau$, such that: $\theta = \tau \circ \sigma$.
\end{definition} % TODO: andere namen? maybe

For example, for the problem $\alpha x = zx$, both $\sigma_1 = [z/\alpha]$ and $\sigma_2 = [\lambda y.zx, \lambda y.zx/ \alpha, z]$ are solutions, %TODO: zeigen warum?
but $\sigma_1$ is more general than $\sigma_2$, since there exists a substitution $\sigma_3 = [\lambda y.zx/z]$, and:
\begin{align*}
    \sigma_3 \circ \sigma_1 = [z[\lambda y.zx/z], \lambda y.zx /\alpha, z] = [(\lambda y.zx), \lambda y.zx /\alpha, z] = \sigma_2 %TODO: wirklich? 
\end{align*}
%% TODO: convert to ND


\subsection{Conversion}


\begin{definition}[Beta-Conversion]
    A single step of beta-conversion $e_1 \betaconv e_2$ is defined as follows:
    \begin{align*}
        \constructor.\textbf{case}\{\dots,\exconstructor{\overline{x}}\Rightarrow e\} 
        \betaconv & e[\overline{e}/x] \tag{$\beta$-Data}\\
        \excopatmatch{\dots,d(\overline{x}) \Rightarrow e, \dots}.d(\overline{e})
        \betaconv & e[\overline{e}/x]  \tag{$\beta$-Codata}
    \end{align*}
    We require that the constructor $K(\overline{e})$ and the constructor $\exconstructor{\overline{x}}$ have the same number of arguments.
    This, in short ensures that we don't generate stuck terms, i.e. Terms that can't be evaluated. %TODO: warum hier und nicht in der syntaxdefinition?
\end{definition}

\begin{definition}[Eta-Conversion for Codata]
    A single step of eta-conversion $e_1 \etaconv e_2$ is defined as follows:
    \begin{align*}
        \excopatmatch{d(\overline{x})\Rightarrow e.d(\overline{x})} 
        \etaconv & e \quad (\text{if } \overline{x}\notin \FV{e}) \tag{$\eta$-Codata}
    \end{align*}
\end{definition}


\section{The Unification Problem}

The Unification Problem is described by a set of equations with expressions on each side $\overline{e=e}$ containing unknown unification variables $\unifvar$,
where our goal is to find a simultaneous substitution $[\listofexpr / \unifvar_1, \dots, \unifvar_n]$ which substitutes expressions for unification variables, 
such that the sides of the given equations are the same (respectively). 
\\ %TODO: hier weniger formell/nochmal umformulieren?
To describe this formally, we need to expand our syntax to be able to represent unification variables.

\begin{table}[!h]
    \centering
        \begin{tabular}{L C L l}
            e & ::= & \unifvar & Unification variable \\
              & | & x & Variable \\
              & | & \constructor & Constructor \\
              & | & \destructor & Destructor\\
              & | & \patmatch & Pattern match\\
              & | & \copatmatch & Copattern match
        \end{tabular}
\end{table}

In unification, we are not concerned with syntactic equality, but want a broader set of terms to be equal to one another.
Depending on the type of unification problem one wants to solve, they may want to only include beta-equality or both 
beta- and eta-equality.

Thus, I will use $\equiv$ for unification problems, where two terms are either syntactically equivalent, beta-equivalent or eta-equivalent.
This essentially means that two terms are equivalent if they are equivalent after function application and/or are equivalent externally.

\begin{definition}[(Higher-Order) Unification Problem]
    A unification problem is a set of equations consisting of expressions $\overline{e \equiv e}$ consisting of our expanded syntax.
\end{definition}

\begin{definition}[Solution]
    A solution to a given unification problem is described by a simultaneous substitution $[\listofexpr / \unifvar_1, \dots, \unifvar_n]$
    which when applied to the problem solves it, i.e. makes the sides of the equations equal %TODO: formellere formulierung?
\end{definition}


For some unification problems, there exists a solution and it is obvious:
The solution to $\unifvar \equiv True$ is $[True/\unifvar]$

For some unification problems, it is rather obvious that there is no solution on the other hand:
There is no mapping of unification variables that make both sides of $True \equiv False$ the same.

Solutions aren't necessarily unique, either. The problem $\unifvar.\texttt{Ap}(5) \equiv 5$ has multiple solutions: 
%$\unifvar \mapsto \lambda x.x$ and $\unifvar \mapsto \lambda x.5$. %TODO: are these actually the only solutions?

$[\excopatmatch{\texttt{Ap}(x)\Rightarrow x} / \unifvar]$ and $[\excopatmatch{\texttt{Ap}(x)\Rightarrow 5} / \unifvar]$,\\
where \texttt{Ap} is a function applicator. 
Thus, the two given solutions are the identity function and the constant function 5, which when consuming the argument 5, both give back 5.

The interesting unification problems are those where it is not clear at first sight whether there exists a solution.
%TODO: die formulierung ist ekelhaft




\subsection{Types Of Solutions} %TODO: besserer name?

To be able to discuss the algorithm for solving the unification problem, we need to define some helpful concepts first.
    

\begin{definition}[Most General]
    A solution is the most general unifier (mgu), iff. it is more general than all other solutions.
\end{definition} 

There does not always exist a mgu, for instance in the problem provided above ($\unifvar.\texttt{Ap}(5) \equiv 5$) %TODO: lieber dem bsp einen namen geben? oder noch eine hyperref?

\subsection{Fundamental Results}

\begin{definition}[First-Order unification]
To better understand higher-order unification, we first want to take a look at some simpler unification problems.
First-order unification is pretty limited, but in many applications all we need, as in most type checking.

\begin{table}[!h]
    \centering
        \begin{tabular}{L C L l}
            e, t & ::= & \unifvar & Unification variable \\
                & | & x & Variable \\
                & | & \constructor & Constructor \\
        \end{tabular}
\end{table}
\end{definition}

For example: $Int\rightarrow \unifvar \equiv Int \rightarrow Bool$ is a first-order unification problem on types which has the solution 
$[Bool/\unifvar]$. 
Note that we are dealing with type constructors, i.e. the type $Int\rightarrow Bool$ would be constructed as $Type(Int, Bool)$ where $Type(\cdot)$ is a constructor.

On the other hand, $\unifvar\equiv List(\unifvar)$ should not have a solution. This is the negative example for the occur check rule below.

Also note that even though we are still using $\equiv$ in equations, in practice this amounts to syntactic equality, since we can't have any beta- or eta- redexes. 

\begin{theorem}[Decidability of First-Order Unification]
    For first-order unification, there exists an algorithm on equations $E$, which always terminates, and returns the solution if there exists one. 
    In particular, this solution is always a mgu (i.e. if there is a solution, then there always exists a most general one).
\end{theorem}

\begin{definition}[Unification algorithm for First-Order Unification ]
    $\perp$ is the symbol for fail.
    The algorithm is defined by non-deterministically applying the below rules:
    \begin{align*}
        E & \cup \{ e \equiv e\} \Rightarrow E \tag{delete}\\
        E & \cup \{\exconstructor{e_1 ... e_n} \equiv \exconstructor{t_1 ... t_n}\} \Rightarrow E \cup \{e_1 \equiv t_1, ... e_n \equiv t_n\}\tag{decompose}\\
        E & \cup \{K_1(e_1,...e_n) \equiv K_2(t_1, ..., t_m)\}  \Rightarrow \perp \quad\text{if $K_1 \neq K_2$ or if $n\neq m$} \tag{conflict}\\
        E & \cup \{e \equiv \unifvar\} \Rightarrow E \cup \{\unifvar \equiv e\}\tag{swap}\\ 
        E & \cup \{\unifvar \equiv e\} \Rightarrow E[e/\unifvar] \cup \{\unifvar \equiv e\} \quad\text{if $\unifvar \in E$ and $\unifvar \notin e$}  \tag{eliminate}\\
        E & \cup \{\unifvar \equiv K(e_1,...e_n)\}\Rightarrow \perp \quad \text{if $\unifvar \in e_1,...e_n$} \tag{occurs check} 
    \end{align*}
\end{definition} 
%TODO: will ich die regeln umsortieren?

This algorithm is based on the version presented by Martelli and Montanari in \cite{10.1145/357162.357169},
adapted to our syntax.


\begin{theorem}[Decidability of Higher-Order Unification]
    Higher-order unification includes unification problems containing higher-order terms (equivalent to lambda abstractions),
    and is covered by our introduced syntax. %TODO: bessere formulierung?
    Higher-order unification is not decidable. This can be proven through reducing Hilbert's tenth problem to the unification problem.
\end{theorem}

\section{Pattern Unification}

Pattern Unification, also sometimes called the Pattern Fragment %TODO: oder anders rum?
is a subsection of Higher-Order Unification. It can be seen as an extension of first-order unification
with its solution being similarly simple as the one to first-order unification.
It was described first by Miller in \cite{10.1093/logcom/1.4.497}.
Since we are dealing with what amounts to an extension to the lambda calculus (as our syntax also contains constructors), we need to extend our definition to more than just function applications. 
This means that (in practice,) the pattern fragment described by Miller is a subset of our definition.

\begin{definition}[Pattern]
    A pattern is any term $p$ 
    \begin{align*}
        p ::= \unifvar \partition p.d(\listofvar)
    \end{align*}
    where it holds that all $\listofvar$ are distinct variables.
\end{definition}
%The intuition here is that in the above case, $\unifvar$ is an unknown function applied to its arguments $\listofexpr$, 
%and anytime we apply an unknown function, we should read the problem as a definition for the function.
%TODO: mehr erklärung?

If You are familiar with other definitions of patterns, You might have noticed that 
we have omitted the part about the variables being bound. We require this later on when looking at patterns as subterms.

%TODO: beispiele mit anderem destruktor??

\begin{theorem}[Decidability of Pattern Unification]
    Patterns Unification is decidable. If there exists a solution, there also exists a mgu.
\end{theorem}

Note that first-order unification is not a subset of pattern unification. 
Equations containing first-order terms that don't contain patterns still remain solvable.
%TODO: ist das offensichtlich? bzw soll ich das in der einleitung klar machen oder besser beschreiben?


To be able to talk about the algorithms for solving unification problems, we need some another definition: %TODO: umformulieren
\begin{definition}[Normal Form]
    The normal form \textbf{NF} is defined as follows:
    \begin{align*}
        n &::= x \partition \unifvar \partition n.d(\overline{v})\partition n.\textbf{case}\{\overline{K(\overline{x})\Rightarrow v}\}\\
        v &::= n \partition K(\overline{v}) \partition \excopatmatch{\overline{d(\overline{x})\Rightarrow v}}
    \end{align*}
    Terms that satisfy the $n$-definition are called neutral terms, $v$-terms are called values.
\end{definition}
Note that these are the terms that do not contain beta-redexes. 
A beta-redex is a term that can be transformed using a beta-conversion. 
%TODO: examples

\begin{theorem}
    If a unification problem has a solution, that solution is the same for the normal form of that same problem. 
    No normal form of a problem that doesn't have a solution has a solution either.
\end{theorem}

Furthermore, we will only be looking at terms that have a normal form. This is enough in most applications. %TODO: ausschmücken?

This is helpful for us, since we only have to consider normal forms in our algorithm, as in: 
Met with a term, we first reduce it to its normal form, and then solve for our solution.

\section{The Algorithm For Higher-Order Unification}
(Nur weil ich eine Citation irgendwo brauche)
In \cite{DBLP:books/el/RV01/Dowek01}, Huet's algorithm is described. %TODO: spelling?




%%%%%%%%%%%%%%%%%%%%%%%%%%%%%%%%%%%%%%%%%%%%%%%%%%%%%%%%%%%%%%%%%%%%%%%%%%%%%
%%% Bibliography
%%%%%%%%%%%%%%%%%%%%%%%%%%%%%%%%%%%%%%%%%%%%%%%%%%%%%%%%%%%%%%%%%%%%%%%%%%%%%

%\addcontentsline{toc}{chapter}{Bibliography}

\bibliographystyle{abbrv}
\bibliography{mylit}
%% Obige Anweisung legt fest, dass BibTeX-Datei `mylit.bib' verwendet
%% wird. Hier koennen mehrere Dateinamen mit Kommata getrennt aufgelistet
%% werden.

\cleardoublepage

%%%%%%%%%%%%%%%%%%%%%%%%%%%%%%%%%%%%%%%%%%%%%%%%%%%%%%%%%%%%%%%%%%%%%%%%%%%%%
%%% Erklaerung
%%%%%%%%%%%%%%%%%%%%%%%%%%%%%%%%%%%%%%%%%%%%%%%%%%%%%%%%%%%%%%%%%%%%%%%%%%%%%
\thispagestyle{empty}
\section*{Selbständigkeitserklärung}

Hiermit versichere ich, dass ich die vorliegende Bachelorarbeit 
selbständig und nur mit den angegebenen Hilfsmitteln angefertigt habe und dass alle Stellen, die dem Wortlaut oder dem 
Sinne nach anderen Werken entnommen sind, durch Angaben von Quellen als 
Entlehnung kenntlich gemacht worden sind. 
Diese Bachelorarbeit wurde in gleicher oder ähnlicher Form in keinem anderen 
Studiengang als Prüfungsleistung vorgelegt. 

\vskip 3cm

Ort, Datum	\hfill Unterschrift \hfill 


%%% Ende
%%%%%%%%%%%%%%%%%%%%%%%%%%%%%%%%%%%%%%%%%%%%%%%%%%%%%%%%%%%%%%%%%%%%%%%%%%%%%

\end{document}

\iffalse
Für wenn wir patterns in mehr kontext behandeln i guess

Some examples of patterns:
\begin{align*}
    \begin{split}
    t_1 &= \excopatmatch{\Ap{x}\Rightarrow \unifvar.\Ap{x}} \\
    t_2 &= \excopatmatch{\Ap{x}\Rightarrow \excopatmatch{\Ap{y}\Rightarrow (\unifvar.\Ap{y}).\Ap{x}}}    
    \end{split}
\end{align*} 
In each of these terms, our unification variable is applied to distinct variables which are bound through  cocases.

However, these terms are not patterns:
\begin{align*}
    \begin{split}
        t_3 &= (\unifvar.\Ap{x}).\Ap{y}\\
        t_4 &= \excopatmatch{\Ap{y}\Rightarrow (\unifvar.\Ap{x})}\\
        t_5 &= \excopatmatch{\Ap{x}\Rightarrow (\unifvar.\Ap{x}).\Ap{x}}
    \end{split}
\end{align*}
as they all don't contain exactly one cocase per variable which would bind that variable distinctly.
%TODO: macht das sinn mit dem distinctly? 
\fi
