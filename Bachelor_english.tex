%%%%%%%%%%%%%%%%%%%%%%%%%%%%%%%%%%%%%%%%%%%%%%%%%%%%%%%%%%%%%%%%%%%%%%%%%%%%%
%%% LaTeX-Rahmen fuer das Erstellen von englischen Bachelorarbeiten
%%%%%%%%%%%%%%%%%%%%%%%%%%%%%%%%%%%%%%%%%%%%%%%%%%%%%%%%%%%%%%%%%%%%%%%%%%%%%

%%%%%%%%%%%%%%%%%%%%%%%%%%%%%%%%%%%%%%%%%%%%%%%%%%%%%%%%%%%%%%%%%%%%%%%%%%%%%
%%% allgemeine Einstellungen
%%%%%%%%%%%%%%%%%%%%%%%%%%%%%%%%%%%%%%%%%%%%%%%%%%%%%%%%%%%%%%%%%%%%%%%%%%%%%

\documentclass[twoside,12pt,a4paper]{article}
%\usepackage{reportpage}
\usepackage{epsf}
\usepackage{graphics, graphicx}
\usepackage{latexsym}
\usepackage[margin=10pt,font=small,labelfont=bf]{caption}
\usepackage[utf8]{inputenc}

\usepackage{amsmath}
\usepackage{array} %for the \newcolumntype macro
\usepackage{amsthm} %for theorems/definitions
\usepackage{thmtools} %for listofheorems TODO: remove/integrate?


\textwidth 14cm
\textheight 22cm
\topmargin 0.0cm
\evensidemargin 1cm
\oddsidemargin 1cm
%\footskip 2cm
\parskip0.5explus0.1exminus0.1ex

% Kann von Student auch nach pers\"onlichem Geschmack ver\"andert werden.
\pagestyle{headings}

\sloppy

\begin{document}


%%%%% Macros


%Macros for Theorems/Definitions/...
\theoremstyle{definition}
\newtheorem{definition}{Definition}[section]

%Macros for possible terms in ND:
\newcommand{\constructor}{K(\overline{e})}
\newcommand{\destructor}{e.d(\overline{e}) }
\newcommand{\patmatch}{e.\textbf{case}\{\overline{K(\overline{x})\Rightarrow e}\}} 
\newcommand{\copatmatch}{\textbf{cocase}\{\overline{d(\overline{x})\Rightarrow e}\}}

%Macros for terms but for examples and such
\newcommand{\exconstructor}[1]{K(#1)}
\newcommand{\exdestructor}[1]{e.d(#1) }
\newcommand{\expatmatch}[1]{e.\textbf{case}\{#1\}} 
\newcommand{\excopatmatch}[1]{\textbf{cocase}\{#1\}}

%Macros for structural/technical latex stuff
\newcolumntype{L}{>{$}l<{$}} % a left-aligned column in mathmode
\newcolumntype{C}{>{$}c<{$}} % a center-aligned column in mathmode

\newcommand{\unifvar}{\alpha^?}
\newcommand{\FV}[1]{\text{FV}(#1)}
\newcommand{\listofexpr}{e_1,...e_n}
\newcommand{\listofvar}{x_1,...x_n}
\newcommand{\betaconv}{\equiv_{\beta}^1}
\newcommand{\etaconv}{\equiv_{\eta}^1}

%%%%%%%%%%%%%%%%%%%%%%%%%%%%%%%%%%%%%%%%%%%%%%%%%%%%%%%%%%%%%%%%%%%%%%%%%%%%
%%% Layout Title page
%%%%%%%%%%%%%%%%%%%%%%%%%%%%%%%%%%%%%%%%%%%%%%%%%%%%%%%%%%%%%%%%%%%%%%%%%%%%
 
\begin{titlepage}
 \begin{center}
  {\LARGE Eberhard Karls Universit\"at T\"ubingen}\\
  {\large Mathematisch-Naturwissenschaftliche Fakultät \\
Wilhelm-Schickard-Institut f\"ur Informatik\\[4cm]}
  {\huge Bachelor Thesis Bioinformatics\\[2cm]}
  {\Large\bf  Title of thesis\\[1.5cm]}
 {\large Name}\\[0.5cm]
Date\\[3cm]
\begin{center}
{\small\bf Reviewer}\\[0.5cm]
 {\large Name Reviewer}\\
  {\footnotesize Department of Computer Science\\
	University of T\"ubingen}
  \end{center}
	
\begin{center}
{\small\bf Supervisor}\\[0.5cm]
  {\large Name Supervisor}\\
  {\footnotesize Address\\
	University of T\"ubingen}\end{center}

  \end{center}
\end{titlepage}
%%%%%%%%%%%%%%%%%%%%%%%%%%%%%%%%%%%%%%%%%%%%%%%%%%%%%%%%%%%%%%%%%%%%%%%%%%%%
%%% Layout back of title page
%%%%%%%%%%%%%%%%%%%%%%%%%%%%%%%%%%%%%%%%%%%%%%%%%%%%%%%%%%%%%%%%%%%%%%%%%%%%

\thispagestyle{empty}
\vspace*{\fill}
\begin{minipage}{11.2cm}
\textbf{Name, first name:}\\
\emph{Title of thesis}\\ Bachelor Thesis Bioinformatics\\
Eberhard Karls Universit\"at T\"ubingen\\
Period: from-till
\end{minipage}
\newpage

%%%%%%%%%%%%%%%%%%%%%%%%%%%%%%%%%%%%%%%%%%%%%%%%%%%%%%%%%%%%%%%%%%%%%%%%%%%%

\pagenumbering{roman}
\setcounter{page}{1}

%%%%%%%%%%%%%%%%%%%%%%%%%%%%%%%%%%%%%%%%%%%%%%%%%%%%%%%%%%%%%%%%%%%%%%%%%%%%
%%% Abstract
%%%%%%%%%%%%%%%%%%%%%%%%%%%%%%%%%%%%%%%%%%%%%%%%%%%%%%%%%%%%%%%%%%%%%%%%%%%%

\section*{Abstract}

Write here your abstract.
\newpage
%%%%%%%%%%%%%%%%%%%%%%%%%%%%%%%%%%%%%%%%%%%%%%%%%%%%%%%%%%%%%%%%%%%%%%%%%%%%
%%% Acknowledgements
%%%%%%%%%%%%%%%%%%%%%%%%%%%%%%%%%%%%%%%%%%%%%%%%%%%%%%%%%%%%%%%%%%%%%%%%%%%%
\section*{Acknowledgements}

Write here your acknowledgements.

\cleardoublepage

%%%%%%%%%%%%%%%%%%%%%%%%%%%%%%%%%%%%%%%%%%%%%%%%%%%%%%%%%%%%%%%%%%%%%%%%%%%%%
%%% Table of Contents
%%%%%%%%%%%%%%%%%%%%%%%%%%%%%%%%%%%%%%%%%%%%%%%%%%%%%%%%%%%%%%%%%%%%%%%%%%%%%

\renewcommand{\baselinestretch}{1.3}
\small\normalsize

\tableofcontents

\renewcommand{\baselinestretch}{1}
\small\normalsize

\listoftheorems

\cleardoublepage

%%%%%%%%%%%%%%%%%%%%%%%%%%%%%%%%%%%%%%%%%%%%%%%%%%%%%%%%%%%%%%%%%%%%%%%%%%%%%
%%% Main Part
%%%%%%%%%%%%%%%%%%%%%%%%%%%%%%%%%%%%%%%%%%%%%%%%%%%%%%%%%%%%%%%%%%%%%%%%%%%%%

\pagenumbering{arabic}
\setcounter{page}{1}

\section{Introduction}

%TODO

\section{The Untyped Calculus ND}

I will be introducing the Untyped Calculus ND, based on ... %TODO citation

\subsection{Syntax of the Untyped Calculus ND}

%TODO: introduction

%% explanation of lists

Some knowledge of notation is necessary to familiarize oneself with the syntax of the Untyped Calculus.
$X$ represents a (possibly empty) sequence $X_1, ... X_i, ... X_n$.  
%TODO: das mit dem simultaneos indexing besser verstehen
%TODO: das mit dem X_1, ... 
%TODO: will ich variablen und co auch hier einführen?
\\
A pattern match $e.\textbf{case}\{\overline{K(\overline{x})\Rightarrow e}\}$ 
matches a term $e$ against a sequence of clauses, each clause consisting of a constructor and an expression.
The expression associated with first constructor %TODO: oder kann es nur eine geben?  
that matches the term is the result of the pattern match. %TODO: kann man hier überhaupt von result sprechen?
For a copattern match $\textbf{cocase} \{\overline{d(\overline{x}) \Rightarrow e}\}$
the same rules apply, but instead of constructors, the term is matched against destructors.

\iffalse
\begin{equation}
    \begin{split}
        e ::&= x \quad \text{Variables} \\
        &| \; \constructor       \quad \text{Constructors} \\
        &| \; \destructor         \quad \text{Destructors} \\
        &| \; \patmatch       \quad \text{Pattern match}\\
        &| \; \copatmatch      \quad \text{Copattern match}
    \end{split}
\end{equation}
\fi

\begin{definition}[Terms of the Calculus ND]

\begin{table}[!h]
\centering
    \begin{tabular}{L C L l}
        e & ::= & x & Variable \\
          & | & \constructor & Constructor \\
          & | & \destructor & Destructor\\
          & | & \patmatch & Pattern match\\
          & | & \copatmatch & Copattern match
    \end{tabular}
\end{table}
\end{definition}



In pattern and copattern matches, every con- or destructor may occur no more than once.
%TODO: erklären warum? => verstehen warum (das reicht)?

Let's look at some examples to ... %TODO: Formulierung 
All rudimentary datatypes can be constructed through constructors: %TODO: zu blanket statement/stimmt das wirlich? 
\\
The Booleans \texttt{True} and \texttt{False} are constructors on an empty sequence.\\
%TODO: integers müssen wir genau so machen, oder?
%TODO: gibt es noch andere interessante/einleuchtende beispiele?
%TODO: Idee für Strings: wir machen das gleiche mit buchstaben wie bei booleans 
%      und representieren dann strings als konstruktoren die eine liste an buchstaben nehmen?

Abstract data types like lists, arrays, records etc. can similarly be defined through constructors:\\
\texttt{Cons(True, Cons(False, Nil))} and \texttt{Date(...)} %TODO: wie werden nummern repräsentiert?

We can use Pattern matching to represent conditionals:
Lik this if statement: "$\textbf{if} e_1 \textbf{then} e_2 \textbf{else} e_3$" which is analogous to:
$e_1.\textbf{case}\{\texttt{True}\Rightarrow e_2, \texttt{False}\Rightarrow e_3\}$
Or the expression which tests whether a given list contains the number 5:
$e.\textbf{case}\{\}$

\subsection{Free Variables, Substitutions, Contexts}

\begin{definition}[Free Variables]
    The set of free variables of a term $e$ is FV($e$). A term is closed if this set is empty.
    Free Variables are defined recursively over the structure of terms as follows:
    \begin{align*}
        \begin{split}
            \FV{x} &:= \{x\}\\
            \FV{\exconstructor{\listofexpr}} &:= \FV{e_1} \cup \dots\cup \FV{e_n}\\
            \FV{\exdestructor{e_1,...e_n}} &:= \FV{e}\cup FV(e_1) \cup \dots \cup \FV{e_n}\\
            \FV{\patmatch} &:= \FV{e}\cup (\FV{e_1}\backslash\overline{x})\cup \dots\cup (\FV{e_n}\backslash\overline{x})\\
            \FV{\copatmatch} &:= (\FV{e_1}\backslash\overline{x})\cup \dots\cup (\FV{e_n}\backslash\overline{x})
        \end{split}
    \end{align*}
%TODO: explanation
\end{definition}


\begin{definition}[Substitution]
    A simultaneous substitution $\sigma$ of the terms $\listofexpr$ for the distinct variables $\listofvar$ is defined as follows:
    \begin{align*} %TODO: geht das mit dem distinct da oben unter?
        \sigma ::= [\listofexpr \backslash\listofvar]
    \end{align*}
\end{definition}

The set of variables for which the substitution is defined is called the domain.
The set of free variables which appear in the substitution is called the range. %TODO: formulierung ausschmücken?

\begin{definition}[Domain and Range of a Substitution]
    The definitions of Domain and Range of a Substitution are as follows:
    \begin{align*}
        \texttt{dom}([\listofexpr/\listofvar]) &:= \{\listofvar\}\\
        \texttt{rng}([\listofexpr/\listofvar]) &:= \FV{e_1}\cup \dots\cup\FV{e_n}
    \end{align*}
\end{definition}

What is actually interesting is what happens when we appy a substitution to an expression

\begin{definition}[Action of a Substitution]
    The action of a substitution $\sigma$ on a term $e$, written as $e \sigma$ and is defined as follows:
    \begin{align*}
        x[\listofexpr / \listofvar] :=& e_i \quad(\text{if } x=x_i) \\
        y\sigma :=& y\quad (\text{if }y\notin \texttt{dom}(\sigma))\\
        (\exconstructor{\listofexpr})\sigma :=& \exconstructor{e_1 \sigma,\dots , e_n \sigma}\\
        (\exdestructor{\listofexpr}) \sigma:=& (e \sigma).d(e_1 \sigma,\dots , e_n \sigma)\\
        (\patmatch)\sigma:=& (e\sigma).\textbf{case}\{\overline{K(\overline{y}\Rightarrow (e\sigma')\sigma)}\}\\
        (\copatmatch)\sigma:=& \excopatmatch{\overline{(d(\overline{y})\Rightarrow(e\sigma')\sigma)}}
    \end{align*} 
    Where $\sigma'$ is a substitution that ensures that we don't bind new variables: 
    $\sigma'$ has the form $[y_1,\dots, y_n/\listofvar]$ and all $y_i$ are fresh for both the domain and the range of $\sigma$.
\end{definition}

The composition of two substitutions $\sigma_2 \circ \sigma_1$ which is equivalent to first applying the substitution $\sigma_1$, then the substitution $\sigma_2$.

\begin{definition}[Composition of Subsitutions]
    Given two substitutions
    \begin{align*}
        \sigma_1 := [\listofexpr/\listofvar],\qquad \sigma_2 := [t_1,\dots,t_m/y_1,\dots,y_m]
    \end{align*}
    Composition is defined as:
    \begin{align*}
        \sigma_2\circ\sigma_1 := [e_1\sigma_2,\dots,e_n\sigma_2,t_j,\dots,t_k/\listofvar,y_j,\dots,y_k]
    \end{align*}
    Where $j, \dots, k$ ist the greatest sub-range of indices $1,\dots,m$ such that none of the variables $y_j$ to $y_k$ is in the domain of $\sigma_1$ 
    %TODO: kann man das besser formulieren?
\end{definition}

\subsection{Conversion}


\begin{definition}[Beta-Conversion]
    A single step of beta-conversion $e_1 \betaconv e_2$ is defined as follows:
    \begin{align*}
        \constructor.\textbf{case}\{\dots,\exconstructor{\overline{x}\Rightarrow e}\} 
        \betaconv & e[\overline{e}/x] \tag{$\beta$-Data}\\
        \excopatmatch{\dots,d(\overline{x} \Rightarrow e, \dots)}.d(\overline{e})
        \betaconv & e[\overline{e}/x]  \tag{$\beta$-Codata}
    \end{align*}
    %TODO: Hier verstehe ich nicht was gleich viele argumente haben soll!
\end{definition}

\begin{definition}[Eta-Conversion for Codata]
    A single step of eta-conversion $e_1 \etaconv e_2$ is defined as follows:
    \begin{align*}
        \expatmatch{K(\overline{x})\Rightarrow K(\overline{x})} \etaconv & e \tag{$\eta$-Data}\\
        \excopatmatch{d(\overline{x})\Rightarrow e.d(\overline{x})} 
        \etaconv & e \quad (\text{if } \overline{x}\notin \FV{e}) \tag{$\eta$-Codata}
    \end{align*}
\end{definition}


\section{The Unification Problem}

The Unification Problem is described by a set of equations with expressions on each side $e=e$ with unknown unification variables $\unifvar$,
where our goal is to find a number of pairs/mappings $\unifvar \mapsto e$ consisting of unification variables and expressions, 
such that when substituting the expressions for the unification variables both sides of the 
given equations are the same. %TODO: das als listen darstellen oder lieber nicht?
\\
To describe this formally, we need to expand our syntax to be able to represent unification variables.

\begin{table}[!h]
    \centering
        \begin{tabular}{L C L l}
            e & ::= & \unifvar & Unification variable \\
              & | & x & Variable \\
              & | & \constructor & Constructor \\
              & | & \destructor & Destructor\\
              & | & \patmatch & Pattern match\\
              & | & \copatmatch & Copattern match
        \end{tabular}
\end{table}

For some unification problems, there exists a solution and it is obvious:
The solution to $\unifvar = True$ is $\unifvar \mapsto True$

For some unification problems, it is rather obvious that there is no solution on the other hand:
There is no mapping of unification variables that make both sides of $True = False$ the same.

Solutions aren't necessarily unique, either. The problem $\unifvar 5 = 5$ has multiple solutions: 
$\unifvar \mapsto \lambda x.x$ and $\unifvar \mapsto \lambda x.5$. %TODO: are these actually the only solutions?

The interesting unification problems are those where it is not clear at first sight whether there exists a solution.
%TODO: die formulierung ist ekelhaft

\subsection{Types Of Solutions} %TODO: besserer name

To be able to discuss the algorithm for solving the unification problem, we need to define some helpful concepts first.

\begin{definition}[Idempotency]
    A solution $\sigma$ is idempotent, iff. $\sigma \circ \sigma = \sigma$.
    Concretely, this means that it doesn't matter how often we apply a solution to a given problem.
\end{definition}

% For example, $\alpha \mapsto x$ is a solution for $\lamba\alpha.x = \lambda x.\alpha$
For example, $[\lambda x.x / y]$ is idempotent, since:
\begin{align*}
    [\lambda x.x / y] \circ [\lambda x.x / y] = [(\lambda x.x)[\lambda x.x / y] / y]
    = [\lambda x.x / y] 
\end{align*}
On the other hand, the solution $[\lambda y. x / x]$ is not idempotent, since:
\begin{align*}
    [\lambda y. x / x] \circ [\lambda y. x / x] = [(\lambda y.x)[\lambda y.x / x] / x] 
    = [\lambda yy.x / x] \neq [\lambda y. x / x]
\end{align*}


\begin{definition}[More General] %TODO: oder generality?
    A solution $\sigma$ is more general than a solution $\theta$, iff. there exists a mapping $\tau$, such that: $\theta = \tau \circ \sigma$.
\end{definition} % TODO: andere namen? maybe

For example, for the problem $\alpha x = zx$, both $\sigma_1 = [z/\alpha]$ and $\sigma_2 = [\lambda y.zx, \lambda y.zx/ \alpha, z]$ are solutions, %TODO: zeigen warum?
but $\sigma_1$ is more general than $\sigma_2$, since there exists a substitution $\sigma_3 = [\lambda y.zx/z]$, and:
\begin{align*}
    \sigma_3 \circ \sigma_1 = [z[\lambda y.zx/z], \lambda y.zx /\alpha, z] = [(\lambda y.zx), \lambda y.zx /\alpha, z] = \sigma_2 %TODO: wirklich? 
\end{align*}
    

\begin{definition}[Most General]
    A solution is the most general unifier (mgu), iff. it is more general than all other solutions.
\end{definition}





%%%%%%%%%%%%%%%%%%%%%%%%%%%%%%%%%%%%%%%%%%%%%%%%%%%%%%%%%%%%%%%%%%%%%%%%%%%%%
%%% Bibliography
%%%%%%%%%%%%%%%%%%%%%%%%%%%%%%%%%%%%%%%%%%%%%%%%%%%%%%%%%%%%%%%%%%%%%%%%%%%%%

%\addcontentsline{toc}{chapter}{Bibliography}

%\bibliographystyle{alpha}
%\bibliography{mylit}
%% Obige Anweisung legt fest, dass BibTeX-Datei `mylit.bib' verwendet
%% wird. Hier koennen mehrere Dateinamen mit Kommata getrennt aufgelistet
%% werden.

\cleardoublepage

%%%%%%%%%%%%%%%%%%%%%%%%%%%%%%%%%%%%%%%%%%%%%%%%%%%%%%%%%%%%%%%%%%%%%%%%%%%%%
%%% Erklaerung
%%%%%%%%%%%%%%%%%%%%%%%%%%%%%%%%%%%%%%%%%%%%%%%%%%%%%%%%%%%%%%%%%%%%%%%%%%%%%
\thispagestyle{empty}
\section*{Selbst\"andigkeitserkl\"arung}

Hiermit versichere ich, dass ich die vorliegende Bachelorarbeit 
selbst\"andig und nur mit den angegebenen Hilfsmitteln angefertigt habe und dass alle Stellen, die dem Wortlaut oder dem 
Sinne nach anderen Werken entnommen sind, durch Angaben von Quellen als 
Entlehnung kenntlich gemacht worden sind. 
Diese Bachelorarbeit wurde in gleicher oder \"ahnlicher Form in keinem anderen 
Studiengang als Pr\"ufungsleistung vorgelegt. 

\vskip 3cm

Ort, Datum	\hfill Unterschrift \hfill 


%%% Ende
%%%%%%%%%%%%%%%%%%%%%%%%%%%%%%%%%%%%%%%%%%%%%%%%%%%%%%%%%%%%%%%%%%%%%%%%%%%%%

\end{document}

